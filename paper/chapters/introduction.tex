
\section{Introduction}
\label{introduction}
Most programming languages excel in one specific programming paradigm. While modern programming languages often try to support multiple paradigms, a programming language is still inherently suited best for specific tasks. Modern software products usually have to satisfy a vast range of different requirements. While most requirements can suitably be implemented with a general purpose, multi paradigm language, some requirements need a different approach which lays outside the languages capabilities. One such requirement for example could be logical reasoning. This can best be implemented with a logic programming language such as Prolog. 

Language-oriented programming is a paradigm that tries to bridge the gap between languages of different paradigms. In a language-oriented programming language, programming languages can exist as first-class citizen. A language can be assigned to variables, can be manipulated, and can be used to executed programs. This approach makes it possible to write software which uses vastly different programming paradigms without having to interface between different languages over external interfaces.

LANG-N-PLAY is a functional language-oriented programming language that supports languages as first-class citizens \cite{cimini_effectiveness_2020,cimini_lang-n-play_2018,cimini_languages_2018}. The language was developed by M. Cimini. It was implemented in a combination of OCaml and $\lambda$Prolog, a higher-order logic programming language \cite{miller_programming_2012}. In their paper "On the Effectiveness of Higher-Order Logic Programming in Language-Oriented Programming" \cite{cimini_effectiveness_2020}, Cimini demonstrates on the example of LANG-N-PLAY that higher-order programming languages, such as $\lambda$Prolog, can be a great fit for implementing language-oriented programming languages.

This work is structured as follows. In section \ref{higher-order-logic-programming} we provide an overview of the most important features of higher-order programming languages. Section \ref{features-of-language-oriented-programming} on the other hand gives an overview of possible features of language-oriented programming. Afterwards Section \ref{lang-n-play} showcases how these features are implemented in LANG-N-PLAY. Finally, Section \ref{related-work} gives an overview of work done in the field of language-oriented programming.
